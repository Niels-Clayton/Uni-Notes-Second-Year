\documentclass[a4paper,11pt]{article}
\usepackage[left=2.5cm, right=2.5cm, top=2cm, bottom=2.5cm]{geometry}
\usepackage{graphicx}
\usepackage{amssymb}
\usepackage{amsmath}
\usepackage[modulo]{lineno}

\begin{document}
\title{\LARGE{\textbf{Workshop 2 Homework}}}
\author{Niels Clayton \\300437590}
\maketitle
\hrule

\section{\LARGE{Article Summary}}
\subsection{Summary}
\linenumbers
This article investigates the whether or not species of plants would germinate and live long enough to undergo the first stages of plant development on artificial Martian soil. Three groups of wild plants, crops, and nitrogen fixers were selected to simulate  the viability of growing crops and nitrogen fixing species. Nitrogen fixing species are essential for supplying Martian soil with the required reactive nitrogen to successfully support biomass growth, which will be required for crop growth. It was found that Mars soil stimulant gave the highest biomass when compared to Moon soil, and Earth soil. This shows that it is in theory it's possible to grow plant in Martian  soil, however further research into the affects of gravity, light, as well as the accuracy of the soil is required.

\end{document}
